\hypertarget{inleiding}{%
\section{Inleiding}\label{inleiding}}

Jaarlijks ondergaan er ongeveer 3200 mensen in Nederland alleen al een
beenamputatie, volgens Feenstra en Woods (z.d.). Sommige van deze mensen
krijgen na de beenamputatie een prothese, dit is een kunstmatige
vervanging van een lichaamsdeel. Protheses worden vaak en al lang
gebruikt. Ook wordt er nog veel onderzoek gedaan naar protheses. Zo zijn
de meeste protheses tegenwoordig mechanisch en lijken ze goed op het
oorspronkelijke lichaamsdeel. Een nadeel aan protheses is dat deze vaak
erg duur zijn. De kosten komen door de ontwikkeling die veel tijd in
beslag neemt en er geen massaproductie mogelijk is, omdat protheses op
maat gemaakt moeten worden voor de gebruiker. Het doel van dit
profielwerkstuk om een zo goedkoop mogelijke mechanische handprothese te
ontwikkelen. In dit profielwerkstuk wordt daarom onderzocht of het
mogelijk is om een robotisch aangestuurde 3D-geprinte armprothese te
maken. Er wordt gekeken naar de keuze van materialen, andere manieren
van produceren en een verbeterde algemene werking. Ook zien we graag dat
mensen bij kunnen dragen aan onze prothesehand. Alles wat wij maken zal
dus openbaar en gratis beschikbaar zijn. Allereerst wordt de anatomie
van de hand uitgelegd. Vervolgens kijken we naar enkele bestaande
protheses. Ook worden al bekende materialen en software toegelicht. In
het tweede deel van dit profielwerkstuk wordt het onderzoek beschreven
en daaruit volgt een conclusie. Als laatste wordt gereflecteerd op het
gemaakte profielwerkstuk.
