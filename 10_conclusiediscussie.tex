\hypertarget{conclusie-en-discussie}{%
\section{Conclusie en discussie}\label{conclusie-en-discussie}}

Ons doel was om een goedkopere, makkelijker bewerkbare prothesehand te
ontwikkelen, die mechanisch naar toebehoren werkt. Door de hand te
printen met een 3D-printer, is deze prothesehand vele malen goedkoper
dan prothese handen op de markt van nu. Ook werkt de hand volledig op
aansturing van armspieren. Voor deze prothesehand zijn de volgende
materialen nodig:

\begin{longtable}[]{@{}llll@{}}
\toprule
Product & Aantal & Prijs per stuk (€) & Totale prijs (€)\tabularnewline
\midrule
\endhead
Arduino Nano & 1 & 4,60 & 4,60\tabularnewline
Servo Driver & 1 & 12,66 & 12,66\tabularnewline
SG90 Mini Servo & 5 & 3,75 & 18,75\tabularnewline
Visdraad & 1 & 6,95 & 6,95\tabularnewline
Batterijhouder voor 4x AA & 1 & 4,20 & 4,20\tabularnewline
AA batterijen & 4 & 6,35 (per 4) & 6,35\tabularnewline
MyoWare Muscle Sensor & 1 & 50,95 & 50,95\tabularnewline
Elektrodes & 3 & 1,16 & 3,50\tabularnewline
Jumper wires & 1 & ong 3,50 & 3,50\tabularnewline
3D-printer filament & 1 rol & 22,50 & 22,50\tabularnewline
Totaalprijs prothesehand & - & - & 140,33\tabularnewline
\bottomrule
\end{longtable}

De prothesehand heeft in totaal 140,33 euro gekost. Deze prijs is erg
variabel. Zo kunnen er duurdere motoren worden gebruikt, zodat er meer
kracht beschikbaar is. De keuze en hoeveelheid filament is ook zeer
belangrijk voor de prijs en kan erg variëren. De prijs van de
spiersensor blijft echter zeer constant. Omdat hiervoor niet echt andere
opties zijn, is deze aan de dure kant (bijna één derde van de
totaalprijs).

Wij denken dat dit profielwerkstuk in vele opzichten erg goed heeft
bijgedragen aan ons leren. Niet alleen met het schrijven van verslagen,
maar ook over elektronica, design, 3d-printers en projectmatig werken.
De taakverdeling was erg duidelijk en de communicatie verliep goed. Voor
een vervolg zou de prothesehand complexer kunnen zijn. Denk hierbij aan
meerdere draairichtingen, verbeterde aansturing en versnelde
reactietijd. Bij het produceren van een kunstmatige versie van een van
de ingewikkeldste onderdelen van het lichaam, valt er altijd te
verbeteren. Ook zou de spiersensor zelf kunnen worden ontworpen, wat een
groot verschil in prijs zal opleveren.
